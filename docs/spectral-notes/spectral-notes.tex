\documentclass{article}

%
% Compiler specific packages
%
% Check if using XeTeX or not: https://tex.stackexchange.com/a/54597/39498
\usepackage{ifxetex}
\ifxetex{}
  \usepackage{fontspec}
  \usepackage{csquotes}
  \usepackage{polyglossia}
  \setmainlanguage[variant=british]{english}
\else
  \usepackage[T1]{fontenc}
  \usepackage[utf8]{inputenc}
  \usepackage[UKenglish, english]{babel}
  \usepackage{lmodern}
\fi

%
% Standard packages
\usepackage{xcolor}

%
% Math
\usepackage{amsmath,amsfonts,amssymb,amsthm}
\usepackage{mathtools}
\usepackage{commath}

%
% Hyper-links
\usepackage{hyperref}
\hypersetup{colorlinks,
            linkcolor={red!50!black},
            citecolor={blue!50!black},
            urlcolor={blue!80!black}}

%
% Biblatex
\usepackage[
  backend=biber,
  url=false]{biblatex}
% \addbibresource{BIBLIOGRAPHY.bib}

\theoremstyle{definition}
\newtheorem{definition}{Definition}[section]
\newtheorem{theorem}{Theorem}[section]


\title{Spectral graph theory notes}
\author{John Reid}



\begin{document}
\maketitle



\section{Notation}

We follow Michael Mahoney's Spectral Graph Theory notes for notation.  Let $G =
(V, E, W)$ be an undirected, weighted graph.

\theoremstyle{definition}
\begin{definition}{$G$ is said to be \emph{d-regular} if all its vertices have degree $d$}
\end{definition}


\subsection{Graph matrices}

\theoremstyle{definition}
\begin{definition}{Adjacency matrix, $A \in \mathbb{R}^{n \times n}$}
  \begin{align*}
    A_{ij} &=
    \begin{cases}
      W_{ij} & \textnormal{if } (ij) \in E \\
      0      & \textnormal{otherwise}
    \end{cases}
  \end{align*}
\end{definition}


\theoremstyle{definition}
\begin{definition}{Diagonal degree matrix, $D \in \mathbb{R}^{n \times n}$}
  \begin{align*}
    D_{ij} &=
    \begin{cases}
      \sum_k W_{ik} & \textnormal{if } i = j \\
      0             & \textnormal{otherwise}
    \end{cases}
  \end{align*}
\end{definition}


\theoremstyle{definition}
\begin{definition}{Diffusion operator}
  \begin{align*}
    D^{-1}A
  \end{align*}
\end{definition}


\theoremstyle{definition}
\begin{definition}{Laplacian matrix \emph{(or combinatorial Laplacian matrix)}}
  \begin{align*}
    L = D - A
  \end{align*}
\end{definition}


\theoremstyle{definition}
\begin{definition}{Normalised Laplacian matrix}
  \begin{align*}
    \mathcal{L} = D^{-\frac{1}{2}} L D^{-\frac{1}{2}} = I - D^{-\frac{1}{2}} A D^{-\frac{1}{2}}
  \end{align*}
\end{definition}



\subsection{Graph cuts}

\theoremstyle{definition}
\begin{definition}
  A \emph{cut} $C = (S, T)$ of a graph is a partition of the vertex set $V$.
\end{definition}

\theoremstyle{definition}
\begin{definition}
  An \emph{$s$-$t$-cut} of $G$ is a cut s.t. $s \in S$ and $t \in T$.
\end{definition}

\theoremstyle{definition}
\begin{definition}
  The \emph{cut set} of a cut $(S, T)$ is the set of edges between $S$ and $T$.
\end{definition}

\theoremstyle{definition}
\begin{definition}
  The \emph{sparsity} of a cut $(S, \bar{S})$ of a $d$-regular graph $(V, E)$ is
  \begin{align*}
    \sigma(S) = \frac{|V| E(S, \bar{S})}{d|S||\bar{S}|}
  \end{align*}
\end{definition}

\theoremstyle{definition}
\begin{definition}
  The \emph{edge expansion} of a cut $(S, \bar{S})$ of a $d$-regular graph $(V, E)$ is
  \begin{align*}
    \phi(S) = \frac{E(S, \bar{S})}{d|S|}
  \end{align*}
\end{definition}

It is a fact that $\phi(S)$ and $\sigma(S)$ can only differ by a factor less than or equal to 2.

\theoremstyle{definition}
\begin{definition}
  The \emph{sparsity} of a $d$-regular graph $G = (V, E)$ is
  \begin{align*}
    \sigma(G) = \min_{S \subset V: S \ne \emptyset, S \ne V} \sigma(S)
  \end{align*}
\end{definition}

\theoremstyle{definition}
\begin{definition}
  The \emph{edge expansion} of a $d$-regular graph $G = (V, E)$ is
  \begin{align*}
    \phi(G) = \min_{S \subset V: |S| \le \frac{|V|}{2}} \phi(S)
  \end{align*}
\end{definition}




\subsection{Graph properties}

\theoremstyle{definition}
\begin{definition}
  Let $\lambda_1, \dots, \lambda_n$ be the eigenvalues of a matrix $A$, then the \emph{spectral radius}
  $\rho_A = \rho(A) = \max_i(|\lambda_i|)$.
\end{definition}



\section{Results}

\subsection{Spectral analysis}

\begin{theorem}[Perron-Frobenius-like for regular graphs]
  Let $G$ be a $d$-regular undirected graph and let $\lambda_1 \le \lambda_2 \le \dots \le \lambda_n$ be the
  real eigenvalues of its normalised Laplacian $\mathcal{L} = I - \frac{1}{d}A$, then:
  \begin{enumerate}
    \item $\lambda_1 = 0$ and the associated eigenvector
      $x_1 = \frac{1}{\sqrt{n}} = (\frac{1}{\sqrt{n}}, \dots, \frac{1}{\sqrt{n}})$.
    \item $\lambda_2 \le 2$.
    \item $\lambda_k = 0$ if and only if $g$ has at least $k$ connected components.
    \item $\lambda_n = 2$ if and only if one component is bipartite.
  \end{enumerate}
\end{theorem}

\begin{theorem}[Similar but for adjacency matrix]
  Let $G$ be a $d$-regular undirected graph and let $\alpha_1 \ge \alpha_2 \ge \dots \ge \alpha_n$ and
  $v_1, \dots, v_n$ be the real eigenvalues and respective eigenvectors of its adjacency matrix $A$, then:
  \begin{enumerate}
    \item $\alpha_1 = d$ and $v_1 = \frac{1}{\sqrt{n}} = (\frac{1}{\sqrt{n}}, \dots, \frac{1}{\sqrt{n}})$.
    \item $\alpha_n \ge -d$.
    \item $G$ is connected if and only if $\alpha_1 > \alpha_2$.
    \item $G$ is bipartite if and only if $\alpha_1 = -\alpha_n$.
  \end{enumerate}
\end{theorem}

\begin{theorem}[Basic Cheeger's Inequality]
  Let $G$ be a $d$-regular undirected graph and let $\lambda_1 \le \lambda_2 \le \dots \le \lambda_n$ be the
  real eigenvalues of its normalised Laplacian $\mathcal{L} = I - \frac{1}{d}A$, then:
  \begin{align*}
    \frac{\lambda_2}{2} \le \phi(G) \le \sqrt{2 \lambda_2}
  \end{align*}
\end{theorem}
Note that this directly implies
\begin{align}
  \frac{{\phi(G)}^2}{2} \le \lambda_2 \le 2 \phi(G)
\end{align}


% \printbibliography{}
\end{document}
